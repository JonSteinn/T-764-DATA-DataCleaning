Almost all Icelandic names end with ``son'' or ``dóttir'' meaning son and daughter respectively. Instead of surnames we append those endings (depending on sex) to the first name of a parent (usually the father). Sharing such an ending (especially for females) can skew a similarity function greatly.

There are various ways of measuring similarity between words. One such method is the \emph{Levenshtein distance} \cite{lev} which measures how many character insertions, deletions or substitutions are required to change one word into another. The more similar the words the less the Levenshtein distance. If the strings are equal, no operations are needed. We can remove excess characters from the longer word and replace every remaining character to match the other word, thus the length of the longer word is an upper bound on Levenshtein distance. Let $\omega_1$ and $\omega_2$ be two non-empty words of length $|\omega_1|$ and $|\omega_2|$ respectively and let $\lev{\omega_1}{\omega_2}$ be their Levenshtein distance. We can then map the similarity into probability with $\plev{\omega_1}{\omega_2} = 1-\frac{\lev{\omega_1}{\omega_2}}{\max\{|\omega_1|,|\omega_2|\}}$.